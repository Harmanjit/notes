\chapter{Quantification of ppGpp}

\section{Biochemical Quantification}

PyDPA is a small molecule that fluoresces specifically in the presence of ppGpp.
It uses pyrene-excimer fluorescence that can be measured at 470 nm. 
The fluorescence increases to 17 and 14 fold in presence of ppGpp and pppGpp, respectively.
This does not excite in present of other phosphorylated nucleotides at 470 nm.

This method was compared with the HPLC method and was found to be more sensitive. 

They checked ppGpp levels in E. coli grown on M9 media. Bacteria were treated with Serine hydroxamate, a proxy for amino acid starvation. 
Post 10 minutes of treatment, the cells were spun down, and cold methanol was added and the culture shaken vigorously for 50 sec; the supernatant was held at -80$^o$.

This method has been used multiple times. 

\begin{itemize}
	\item For analysis of stringent response in S. aureus, using very similar protocol.
	\item
\end{itemize}

The issue seems to be that PyDPA seems to interact with inorganic polyphosphate (PPi).
PPi quenches the fluorescence of PyDPA bound to ppGpp as PPi has higher affinity for PyDPA than ppGpp.
This might confound fluorescence values, unless we can show that PPi levels do not change in our control and experimental sample. 